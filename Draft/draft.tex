\documentclass[12pt]{article}
\usepackage[utf8]{inputenc}
\usepackage{amsthm, amssymb, amsmath}
\usepackage{tikz-cd}

\newtheorem{df}{Définition}
\newtheorem{thm}{Théorème}
\newtheorem{lm}{Lemme}
\newtheorem{crl}{Corollaire}
\newtheorem{prop}{Proposition}


\title{Draft filtered colimits}
\author{Sébastien Draux}

\begin{document}

\maketitle

\begin{prop}[Limits in $\mathbf{Set}$]
Let $F : I \to \mathbf{Set}$ then :
$$ \lim F = \{ (x_i)_{i \in I} \  | \ x_i \in F(i)\  \forall i, i' \in I \ \forall f : i \to i' \ F(f)(x_i) = x_{i'}\}$$

with projection map : $p_i(a) = a_i$

\end{prop}

\begin{proof}
Let denote the set above $L$. By definition, the maps $p_i : L \to F(i)$ commute with the morphism of the form $F(f)$.\\
Let $(A, q)$ be a cone of $F$. Let $\varphi : A \to L$ defined by :
$$
\varphi(a) = (q_i(a))_{i \in I}
$$
This map is well-defined because $F(f) \circ q_i = q_{i'}$ by definition.\\
Moreover, it is unique because it has to satisfy $p_i \circ \varphi = q_i$ therefore its coordinate are entirely determined.

$$
\begin{tikzcd}
                                                                                            &  &                                                   &  &  & F(i) \arrow[dd, "F(f)"'] \\
A \arrow[rrrrrd, "q_{i'}", bend right] \arrow[rrrrru, "q_i"', bend left] \arrow[rr, dotted] &  & lim F \arrow[rrru, "p_i"'] \arrow[rrrd, "p_{i'}"] &  &  &                          \\
                                                                                            &  &                                                   &  &  & F(i')                   
\end{tikzcd}
$$

\end{proof}

\begin{prop}[Colimits in $\mathbf{Set}$]
Let $F : J \to \mathbf{Set}$ then :
$$
colim \ F = \left(\bigsqcup_{j \in J} F(j)\right) / \sim
$$
where $\sim$ is the equivalence relation generated by : $\forall j, j' \in J$, $\forall f : j \to j'$, $\forall x \in F(j)$ :
$$
F(f)(x) \sim x
$$
with maps $\tau_j : F(j) \to colim \ F$ such that $\tau_j(x)$ is the class of $x$ in $colim \ F$.
\end{prop}

\begin{proof}
By definition, for $f : j \to j'$ and $x \in F(j)$ we have :
$$
\tau_j(x) = \tau_{j'} (F(f)(x))
$$
hence $\tau_j= \tau_{j'} \circ F(f)$.\\
Let $(A, \sigma)$ a cocone of $F$. Let $\varphi$ defined by :
$$
\varphi(\tau_j(x)) = \sigma_j(x)
$$
This map is well-defined because if we take $f:j \to j'$ then :
$$
\varphi(\tau_{j'}(F(f)(x))) = \sigma_{j'} (F(f)(x)) = \sigma_j(x) = \varphi(\tau_j(x))
$$
because $A$ is a cocone.\\
This map is unique because it is entirely determined by its value of the element of the form $\tau_j(x)$ and has to satisfy $\sigma_j = \varphi \circ \tau_j$.

$$
\begin{tikzcd}
  &  &                                         &  &  & F(j) \arrow[dd, "F(f)"'] \arrow[llllld, "\sigma_j"', bend right] \arrow[llld, "\tau_j"'] \\
A &  & \lim \ F \arrow[ll, "\varphi"', dotted] &  &  &                                                                                          \\
  &  &                                         &  &  & F(j') \arrow[lllllu, "\sigma_{j'}", bend left] \arrow[lllu, "\tau_{j'}"]                
\end{tikzcd}
$$

\end{proof}
sds
\begin{prop}
A category $C$ is filtered if and only if every finite diagram has a cocone.
\end{prop}

\begin{proof}
The implication from right to left is obvious because if a category admits a cocone for every finite diagram, it has has the cocone required to be filtered.\\

Suppose $C$ is filtered. Let $F : J \to C$ a functor with $J$ finite. Let's denote $j_1, \cdots, j_n$ its elements. Using axiom 2 of filtered category, there exist $c \in C$ with morphisms 
\end{proof}

\end{document}
